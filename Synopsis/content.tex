\pdfbookmark{Общая характеристика работы}{characteristic}             % Закладка pdf
\section*{Общая характеристика работы}

\newcommand{\actuality}{\pdfbookmark[1]{Актуальность}{actuality}\underline{\textbf{\actualityTXT}}}
\newcommand{\progress}{\pdfbookmark[1]{Разработанность темы}{progress}\underline{\textbf{\progressTXT}}}
\newcommand{\aim}{\pdfbookmark[1]{Цели}{aim}\underline{{\textbf\aimTXT}}}
\newcommand{\tasks}{\pdfbookmark[1]{Задачи}{tasks}\underline{\textbf{\tasksTXT}}}
\newcommand{\aimtasks}{\pdfbookmark[1]{Цели и задачи}{aimtasks}\aimtasksTXT}
\newcommand{\novelty}{\pdfbookmark[1]{Научная новизна}{novelty}\underline{\textbf{\noveltyTXT}}}
\newcommand{\influence}{\pdfbookmark[1]{Практическая значимость}{influence}\underline{\textbf{\influenceTXT}}}
\newcommand{\methods}{\pdfbookmark[1]{Методология и методы исследования}{methods}\underline{\textbf{\methodsTXT}}}
\newcommand{\defpositions}{\pdfbookmark[1]{Положения, выносимые на защиту}{defpositions}\underline{\textbf{\defpositionsTXT}}}
\newcommand{\reliability}{\pdfbookmark[1]{Достоверность}{reliability}\underline{\textbf{\reliabilityTXT}}}
\newcommand{\probation}{\pdfbookmark[1]{Апробация}{probation}\underline{\textbf{\probationTXT}}}
\newcommand{\contribution}{\pdfbookmark[1]{Личный вклад}{contribution}\underline{\textbf{\contributionTXT}}}
\newcommand{\publications}{\pdfbookmark[1]{Публикации}{publications}\underline{\textbf{\publicationsTXT}}}

\newcommand{\realisation}{\underline{\textbf{\realisationTXT}}}


{\actuality} В последнее время всё более очевидной становится тенденция к повышению точности и чувствительности космических средств, предназначенных для наблюдения и получения информации о положении малоэнергетических целей. Одним из путей решения этой задачи является увеличение размеров оптических систем космического назначения.

Расширение эксплуатационных возможностей такой широкоформатной оптики предполагает, в свою очередь, введение в оптическую систему элементов, позволяющих изменять в пространстве положение визирной оси оптической аппаратуры. Эту задачу можно решить либо поворотом космического аппарата (КА) в пространстве, либо за счет изменения положения одного либо нескольких элементов оптической системы относительно КА. Другим вариантом получения эффекта перенацеливания оптической системы является разворот всей оптической системы относительно КА.
~\autocite{Gosele1999161,Lermontov}

Понятно, что с точки зрения экономии энергии на борту КА наиболее рациональным решением будет перемещение в процессе перенацеливания минимальной массы (одного или нескольких элементов оптической системы). Но, независимо от принятого выбора конструктивного исполнения, при изменении положения перемещаемой массы относительно КА возникнут реактивные силы и моменты, воздействующие на КА, которые приведут к развороту КА вокруг его центра масс в направлении, противоположном направлению перемещения подвижной массы. Таким образом, в результате взаимного перемещения оптической системы (или её элементов) относительно КА на некоторый заданный угол и перемещения самого КА в пространстве ось визирования оптической системы займёт в пространстве некоторое положение, не совпадающее с заданными углами на перенацеливание.  Особенно сильно влияние реактивных моментов и сил в случае инфракрасных оптических систем космического назначения, имеющих значительные габариты массу. 

Как правило, на борту КА функционирует система стабилизации положения КА в пространстве. В результате работы этой системы через некоторое время (значительно большее времени перенацеливания) КА вернётся в положение, которое он занимал до начала процесса перенацеливания. Только тогда ось визирования оптической системы постепенно займёт требуемое положение. Кроме того, на КА закреплено большое количество устройств (антенны, солнечные панели и т.д.), имеющих достаточно низкую частоту собственных колебаний. Если перенацеливание происходит за короткое время, то реактивное воздействие на КА близко по своей природе к ударному воздействию, что может привести к возникновению медленно затухающих собственных колебаний этих устройств относительно КА. Это обстоятельство приводит к возникновению дополнительных гармонических воздействий на КА, что затрудняет работу системы стабилизации КА и приводит к ещё большему затягиванию процесса перенацеливания оси визирования оптической системы.
Следует отметить, что данная тематика в настоящее время разработана недостаточно подробно и имеющийся список научной литературы по данному вопросу весьма скуден.

В связи с этим, актуально проведение подробного исследования результатов влияния реактивных моментов на КА, возникающих в процессе перенацеливания визирной оси оптической системы, входящей в состав КА.


\ifsynopsis
Этот абзац появляется только в~автореферате.
Для формирования блоков, которые будут обрабатываться только в~автореферате,
заведена проверка условия \verb!\!\verb!ifsynopsis!.
Значение условия задаётся в~основном файле документа (\verb!synopsis.tex! для
автореферата).
\else
Этот абзац появляется только в~диссертации.
Через проверку условия \verb!\!\verb!ifsynopsis!, задаваемого в~основном файле
документа (\verb!dissertation.tex! для диссертации), можно сделать новую
команду, обеспечивающую появление цитаты в~диссертации, но~не~в~автореферате.
\fi

% {\progress}
% Этот раздел должен быть отдельным структурным элементом по
% ГОСТ, но он, как правило, включается в описание актуальности
% темы. Нужен он отдельным структурынм элемементом или нет ---
% смотрите другие диссертации вашего совета, скорее всего не нужен.

{\aim} данной является разработка и исследование метода расчёта реактивных моментов, возникающих при перенацеливании оси визирования оптических систем космического назначения, разработка и исследование оборудования для измерения реактивных моментов в наземных условиях, проведение расчётов и экспериментальных исследований конкретных образцов оптических систем космического назначения.

Для~достижения поставленной цели необходимо было решить следующие {\tasks}:
\begin{enumerate}[beginpenalty=10000] % https://tex.stackexchange.com/a/476052/104425
  \item Провести анализ и классификацию существующих методов компенсации реактивных воздействий на КА, возникающих при перенацеливании оси визирования оптической системы.
  \item Разработать методику расчёта средств на борту КА, предназначенных для компенсации реактивных моментов, возникающих при перенацеливании оси визирования для конкретных образцов оптических систем космического назначения.
  \item Разработать испытательный стенд для измерения реактивных воздействий на КА при проведении наземных испытаний оптических систем.
  \item Разработать методику проведения наземных испытаний оптических систем космического назначения с точки зрения измерения реактивных воздействий на КА.
\end{enumerate}


{\novelty}
\begin{enumerate}[beginpenalty=10000] % https://tex.stackexchange.com/a/476052/104425
  \item Впервые проведены расчёты пространственных реактивных воздействий на КА при проектировании нескольких вариантов крупногабаритных оптических систем космического назначения с возможностью перенацеливания визирной оси, которые позволили оптимизировать конструкцию аппаратуры и алгоритм управления устройством перенацеливания.
  \item Разработанная испытательная аппаратура и методика проведения испытаний крупногабаритных оптических систем космического назначения с возможностью перенацеливания визирной оси позволила получить значения допустимых реактивных воздействий, требуемые в техническом задании на проектирование, что подтверждено результатами наземных и лётных испытаний. 
\end{enumerate}

{\influence} \ldots

{\methods} \ldots

{\defpositions}
\begin{enumerate}[beginpenalty=10000] % https://tex.stackexchange.com/a/476052/104425
  \item Первое положение
  \item Второе положение
  \item Третье положение
  \item Четвертое положение
\end{enumerate}
В папке Documents можно ознакомиться с решением совета из Томского~ГУ
(в~файле \verb+Def_positions.pdf+), где обоснованно даются рекомендации
по~формулировкам защищаемых положений.

{\reliability} полученных результатов обеспечивается \ldots \ Результаты находятся в соответствии с результатами, полученными другими авторами.


{\probation}
Основные результаты работы докладывались~на:
перечисление основных конференций, симпозиумов и~т.\:п.

{\contribution} Автор принимал активное участие \ldots

\ifnumequal{\value{bibliosel}}{0}
{%%% Встроенная реализация с загрузкой файла через движок bibtex8. (При желании, внутри можно использовать обычные ссылки, наподобие `\cite{vakbib1,vakbib2}`).
    {\publications} Основные результаты по теме диссертации изложены
    в~XX~печатных изданиях,
    X из которых изданы в журналах, рекомендованных ВАК,
    X "--- в тезисах докладов.
}%
{%%% Реализация пакетом biblatex через движок biber
    \begin{refsection}[bl-author, bl-registered]
        % Это refsection=1.
        % Процитированные здесь работы:
        %  * подсчитываются, для автоматического составления фразы "Основные результаты ..."
        %  * попадают в авторскую библиографию, при usefootcite==0 и стиле `\insertbiblioauthor` или `\insertbiblioauthorgrouped`
        %  * нумеруются там в зависимости от порядка команд `\printbibliography` в этом разделе.
        %  * при использовании `\insertbiblioauthorgrouped`, порядок команд `\printbibliography` в нём должен быть тем же (см. biblio/biblatex.tex)
        %
        % Невидимый библиографический список для подсчёта количества публикаций:
        \phantom{\printbibliography[heading=nobibheading, section=1, env=countauthorvak,          keyword=biblioauthorvak]%
        \printbibliography[heading=nobibheading, section=1, env=countauthorwos,          keyword=biblioauthorwos]%
        \printbibliography[heading=nobibheading, section=1, env=countauthorscopus,       keyword=biblioauthorscopus]%
        \printbibliography[heading=nobibheading, section=1, env=countauthorconf,         keyword=biblioauthorconf]%
        \printbibliography[heading=nobibheading, section=1, env=countauthorother,        keyword=biblioauthorother]%
        \printbibliography[heading=nobibheading, section=1, env=countregistered,         keyword=biblioregistered]%
        \printbibliography[heading=nobibheading, section=1, env=countauthorpatent,       keyword=biblioauthorpatent]%
        \printbibliography[heading=nobibheading, section=1, env=countauthorprogram,      keyword=biblioauthorprogram]%
        \printbibliography[heading=nobibheading, section=1, env=countauthor,             keyword=biblioauthor]%
        \printbibliography[heading=nobibheading, section=1, env=countauthorvakscopuswos, filter=vakscopuswos]%
        \printbibliography[heading=nobibheading, section=1, env=countauthorscopuswos,    filter=scopuswos]}%
        %
        \nocite{*}%
        %
        {\publications} Основные результаты по теме диссертации изложены в~\arabic{citeauthor}~печатных изданиях,
        \arabic{citeauthorvak} из которых изданы в журналах, рекомендованных ВАК%
        \ifnum \value{citeauthorscopuswos}>0%
            , \arabic{citeauthorscopuswos} "--- в~периодических научных журналах, индексируемых Web of~Science и Scopus%
        \fi%
        \ifnum \value{citeauthorconf}>0%
            , \arabic{citeauthorconf} "--- в~тезисах докладов.
        \else%
            .
        \fi%
        \ifnum \value{citeregistered}=1%
            \ifnum \value{citeauthorpatent}=1%
                Зарегистрирован \arabic{citeauthorpatent} патент.
            \fi%
            \ifnum \value{citeauthorprogram}=1%
                Зарегистрирована \arabic{citeauthorprogram} программа для ЭВМ.
            \fi%
        \fi%
        \ifnum \value{citeregistered}>1%
            Зарегистрированы\ %
            \ifnum \value{citeauthorpatent}>0%
            \formbytotal{citeauthorpatent}{патент}{}{а}{}%
            \ifnum \value{citeauthorprogram}=0 . \else \ и~\fi%
            \fi%
            \ifnum \value{citeauthorprogram}>0%
            \formbytotal{citeauthorprogram}{программ}{а}{ы}{} для ЭВМ.
            \fi%
        \fi%
        % К публикациям, в которых излагаются основные научные результаты диссертации на соискание учёной
        % степени, в рецензируемых изданиях приравниваются патенты на изобретения, патенты (свидетельства) на
        % полезную модель, патенты на промышленный образец, патенты на селекционные достижения, свидетельства
        % на программу для электронных вычислительных машин, базу данных, топологию интегральных микросхем,
        % зарегистрированные в установленном порядке.(в ред. Постановления Правительства РФ от 21.04.2016 N 335)
    \end{refsection}%
    \begin{refsection}[bl-author, bl-registered]
        % Это refsection=2.
        % Процитированные здесь работы:
        %  * попадают в авторскую библиографию, при usefootcite==0 и стиле `\insertbiblioauthorimportant`.
        %  * ни на что не влияют в противном случае
        \nocite{vakbib2}%vak
        \nocite{patbib1}%patent
        \nocite{progbib1}%program
        \nocite{bib1}%other
        \nocite{confbib1}%conf
    \end{refsection}%
        %
        % Всё, что вне этих двух refsection, это refsection=0,
        %  * для диссертации - это нормальные ссылки, попадающие в обычную библиографию
        %  * для автореферата:
        %     * при usefootcite==0, ссылка корректно сработает только для источника из `external.bib`. Для своих работ --- напечатает "[0]" (и даже Warning не вылезет).
        %     * при usefootcite==1, ссылка сработает нормально. В авторской библиографии будут только процитированные в refsection=0 работы.
}

При использовании пакета \verb!biblatex! будут подсчитаны все работы, добавленные
в файл \verb!biblio/author.bib!. Для правильного подсчёта работ в~различных
системах цитирования требуется использовать поля:
\begin{itemize}
        \item \texttt{authorvak} если публикация индексирована ВАК,
        \item \texttt{authorscopus} если публикация индексирована Scopus,
        \item \texttt{authorwos} если публикация индексирована Web of Science,
        \item \texttt{authorconf} для докладов конференций,
        \item \texttt{authorpatent} для патентов,
        \item \texttt{authorprogram} для зарегистрированных программ для ЭВМ,
        \item \texttt{authorother} для других публикаций.
\end{itemize}
Для подсчёта используются счётчики:
\begin{itemize}
        \item \texttt{citeauthorvak} для работ, индексируемых ВАК,
        \item \texttt{citeauthorscopus} для работ, индексируемых Scopus,
        \item \texttt{citeauthorwos} для работ, индексируемых Web of Science,
        \item \texttt{citeauthorvakscopuswos} для работ, индексируемых одной из трёх баз,
        \item \texttt{citeauthorscopuswos} для работ, индексируемых Scopus или Web of~Science,
        \item \texttt{citeauthorconf} для докладов на конференциях,
        \item \texttt{citeauthorother} для остальных работ,
        \item \texttt{citeauthorpatent} для патентов,
        \item \texttt{citeauthorprogram} для зарегистрированных программ для ЭВМ,
        \item \texttt{citeauthor} для суммарного количества работ.
\end{itemize}
% Счётчик \texttt{citeexternal} используется для подсчёта процитированных публикаций;
% \texttt{citeregistered} "--- для подсчёта суммарного количества патентов и программ для ЭВМ.

Для добавления в список публикаций автора работ, которые не были процитированы в
автореферате, требуется их~перечислить с использованием команды \verb!\nocite! в
\verb!Synopsis/content.tex!.
 % Характеристика работы по структуре во введении и в автореферате не отличается (ГОСТ Р 7.0.11, пункты 5.3.1 и 9.2.1), потому её загружаем из одного и того же внешнего файла, предварительно задав форму выделения некоторым параметрам

%Диссертационная работа была выполнена при поддержке грантов \dots

%\underline{\textbf{Объем и структура работы.}} Диссертация состоит из~введения,
%четырех глав, заключения и~приложения. Полный объем диссертации
%\textbf{ХХХ}~страниц текста с~\textbf{ХХ}~рисунками и~5~таблицами. Список
%литературы содержит \textbf{ХХX}~наименование.

\pdfbookmark{Содержание работы}{description}                          % Закладка pdf
\section*{Содержание работы}
Во \underline{\textbf{введении}} обосновывается актуальность
исследований, проводимых в~рамках данной диссертационной работы,
приводится обзор научной литературы по~изучаемой проблеме,
формулируется цель, ставятся задачи работы, излагается научная новизна
и практическая значимость представляемой работы.


\underline{\textbf{Первая глава}} посвящена анализу влияния реактивного момента на космический аппарат и методам его измерения.

Реактивные моменты, возникающие в электроприводах систем визирования и сканирования, являются одним из основных источников возмущений, влияющих на стабильность пространственного положения космических аппаратов. В динамических режимах работы — при разгоне, торможении и реверсировании ротора — их величина может многократно превышать уровень возмущений от внешних факторов, что приводит к дрожанию визирной оси и снижению точности ориентации.

Особенно чувствительными к воздействию реактивных моментов оказываются оптико-электронные системы. Даже небольшие угловые колебания визирной оси приводят к заметному снижению пространственного разрешения и к размытию изображения. В оптико-электронных сканирующих системах с приборами с зарядовой связью размытие возникает, когда за время экспозиции фоточувствительный элемент принимает излучение не только от собственной проекции участка поверхности, но и частично от соседних. Это приводит к интегрированию сигналов от разных элементов сцены, снижению резкости, контрастности и читаемости мелких деталей изображения. Характер влияния реактивного момента на изображение можно разделить на несколько составляющих. Низкочастотная компонента связана с медленным изменением угловой скорости и приводит к смещению изображения относительно оптимального положения. Гармоническая компонента возникает из-за колебаний элементов конструкции и резонансов в системе ориентации. Случайная компонента обусловлена нерегулярными импульсами момента при работе приводов и механическими шумами. Все они вызывают размытие и дрожание изображения, а также приводят к снижению частотно-контрастной характеристики и интегральных показателей качества.

Когда траектория смещения изображения в пределах кадра близка к линейной, и снижение частотно-контрастной характеристики (ЧКХ) описывается аппроксимацией
линейного размытия:

\begin{equation*}
	\label{mtf_lf}
	\text{ЧКХ}_{LF}(\nu)=\frac{\sin{\pi \nu L}}{\pi \nu L}
\end{equation*}

где \(L\) "--- величина смещения изображения за время экспозиции, \(\nu\) "--- пространственная частота.

Установлено, что низкочастотные колебания определяют предельное качество изображения. Например при увеличении амплитуды смещения с 10 до 20 мкм значение ЧКХ на частоте Найквиста уменьшается на 41~\%, а при 30 мкм практически падает до нуля~\cite{wahballah2018smear}.

Влияние реактивного момента на качество изображения очевидно, однако для его количественной оценки требуется знание спектральных и амплитудных характеристик возмущающих воздействий. Решение этой задачи невозможно без достоверных методов измерения момента, возникающего в приводах оптико-механических систем и передаваемого на корпус космического аппарата.

Существует множество способов регистрации крутящего момента на валу двигателя, основанных на тензорезистивных, индуктивных, магнитных и оптических принципах. Эти методы позволяют достаточно точно измерять момент на валу двигателя, но они не отражают полной картины реактивного момента, действующего на конструкцию в целом.

Наиболее близким к тематике настоящей работы является метод определения возмущающего момента, реализованный на инерционном стенде для экспериментальной отработки системы наведения антенн космического аппарата \cite{Goncharuk2013}. Функциональная схема устройства представлена на рисунке~\cref{fig:stand}.

\begin{figure}[h!] 
	\centerfloat{
		\includegraphics[scale=0.4]{stand} 
	}
	\legend{1 – система обезвешивания; 2 – штанга; 3 – груз; 4 – датчик угловой скорости с кронштейном для установки; 5 – элемент для передачи момента на выходной вал; 6 – плита установочная; 7 – основание; 8 – объект контроля}
	\caption{Функциональная схема инерционного стенда}
	\label{fig:stand} 
\end{figure}
Конструктивно такой стенд представляет собой установку с приводом и имитатором нагрузки, повторяющим момент инерции антенны, и системой обезвешивания. Реактивный момент определяется по динамике вращения выходного вала на основе зависимости:
\begin{equation*}
	\label{eq:eq_M_disturb}
	M=J_{\text{н}}\cdot \frac{d\omega}{dt},
\end{equation*}
где \(J_{\text{н}}\) "--- момент инерции имитируемой нагрузки, \(\omega\) "--- угловая скорость.

Метод сводиться к регистрации динамики вращения исполнительного механизма и последующему вычислению возмущающего момента через известные параметры имитируемой нагрузки.

Несмотря на практическую ценность, данный метод имеет ряд ограничений: использование имитатора вместо реальной нагрузки, необходимость изготовления отдельных моделей для каждого объекта, локальный характер измерений, а также невозможность анализа ситуаций с одновременной работой компенсирующих устройств. Эти недостатки не позволяют применить его для задач, связанных с оценкой реактивного момента в высокоточных оптико-электронных системах.

Таким образом, анализ существующих решений показал, что известные методы измерения крутящего момента не отражают полную картину динамических процессов в приводах оптико-механических систем и не позволяют определить реактивный момент, действующий на основание прибора.
Выявленные ограничения существующих методов обуславливают необходимость разработки нового подхода, включающего построение математической модели оптико-механической системы и формулировку методики экспериментальных измерений нескомпенсированного реактивного момента.


\underline{\textbf{Вторая глава}} посвящена построению математической модели формирования остаточного реактивного момента в оптико-механической системе и определению способов его снижения за счёт оптимизации профиля разгона исполнительного механизма.

Рассматриваются три объекта: модуль направленного обзора с кардановым подвесом (две степени свободы), оптико-механическое устройство «Зеркало» с редукторными приводами по осям $OY$ и $OZ$, а также оптико-механический сканер для работы в режиме ВЗН. Во первых двух устройствах компенсация реактивного момента реализуется контрвращением маховика, кинематически жёстко связанного с приводом; редукторная связь уменьшает требуемый момент инерции маховика пропорционально квадрату передаточного числа, что позволяет снизить массу и энергоёмкость узла при сохранении компенсирующего воздействия. В сканере компенсация осуществляется отдельным приводом. Такая архитектура позволяет независимо настраивать закон движения зеркала и компенсатора, оптимизируя спектр возмущений и синхронизацию с режимом ВЗН.

Рассмотрим результат воздействия реактивных моментов на основание с учётом расположения центра вращения карданова подвеса оптико-механической системы (ОМС) относительно центра тяжести КА (Рисунок~\cref{fig:tikz_YPK})
\begin{figure}[h!]
	\centerfloat{
		\ifdefmacro{\tikzsetnextfilename}{\tikzsetnextfilename{tikz_example_compiled}}{}% присваиваемое предкомпилированному pdf имя файла (не обязательно)
		\resizebox{0.7\linewidth}{!}{%
		
	\begin{tikzpicture}[scale=1.6, >=stealth, font=\LARGE]
%X0Y0Z0 axis
\draw [->, very thick] (13.75,13.25) -- (13.75,16.25)node[above] {$Y_0$};
\draw [->, very thick] (13.75,13.25) -- (11.5,11.5) node[above left] {$X_0$};
\draw [->,very thick] (13.75,13.25) -- (11.25,13.75) node[above] {$Z_0$};
\node at (13.85, 13.05) {$0_0$};
%Z0 line
\draw [thin, short] (11.25,13.75) -- (16.25,12.75);
%Z line
\draw [thin, short] (13.75,13.25) -- (16.75,13);

%X0Y0Z0 axis proection
\draw [thin] (10,8.25) -- (10,10.75) node[above] {$Y_0$};
\draw [thin] (10,8.25) -- (7.5,8.75) node[above] {$Z_0$};
\draw [thin] (10,8.25) -- (7.5,6.5) node[above left] {$X_0$};


%proection Z0 line
\draw [thin, short] (7.5,8.75) -- (12.5,7.75);
\draw [line width=0.9pt, ->,] (10,8.25) -- (10,10) node[right] {$M_{da}$};
%OXYZ axis
\draw [ ->, very thick] (10,8.25) -- (9.75,6.25) node[left] {$X$};
\draw [->, very thick] (10,8.25) -- (9.32, 10.2) node[above] {$Y$};
\draw[thick,->] (10, 10) arc[start angle=5,end angle=175,x radius=0.3cm, y radius=0.2cm];
\node at (9.7, 10.3) {$\beta$};
\draw [ ->, very thick] (10,8.25) -- (7,8.5) node[above] {$Z$};
\node at (10.20,8.33) {$0$};
\draw[thick,->] (8.8,8.5) arc[start angle=0,end angle=85,radius=-1.2cm];
\node at (9.5, 7.5) {$\alpha$};
%red line
\draw [ color={rgb,255:red,245; green,0; blue,0}, line width=1.6pt, short] (13.75,13.25) -- (10,8.25);
\node at (11, 10)[text=red] {$R$};
\draw [ color=red, line width=0.2pt, ->,] (10,8.25) -- (9.75,9)node[left] {$F_b$};



\draw [line width=0.9pt, ->,] (10,8.25) -- (7.8,8.43) node[below] {$M_{db}$};
\draw [ color=red, line width=0.2pt, ->,] (10,8.25) -- (8.5,8.37) node[below] {$F_a$};
%blue circle
\draw [ color={rgb,255:red,4; green,45; blue,251} , fill={rgb,255:red,45; green,166; blue,240}, line width=0.2pt ] (9.83,7) circle (0.25cm);



\draw [ color=blue, very thick, short] (10,8.25) -- (9.83,7);
\node at (10, 7.5)[text=blue] {$r$};
		\end{tikzpicture}
	}
		
	}
	\legend{}
	\caption[Пример \texttt{tikz} схемы]{Смещение центра масс блока зеркал ОМС}\label{fig:tikz_YPK}
\end{figure}

Математическое описание реактивных моментов учитывает смещение центра масс зеркального блока относительно карданова подвеса на расстояние $r$ и несоосность центра подвеса с центром масс КА $(R_x,R_y,R_z)$. При работе приводов формируются моменты $M_{da}, M_{db}$ и реакции маховиков $M_{ma}, M_{mb}$. Результирующие моменты на осях кардана можно записать в виде:
\begin{equation*}
	M_{ra} = F_a r + M_{da} - M_{ma}, 
	\qquad 
	M_{rb} = F_b r + M_{db} - M_{mb},
\end{equation*}
где $F_a, F_b$ --- силы, действующие на узел зеркал. 

Силы, приложенные к подвесу, проецируются на оси $O_0X_0Y_0Z_0$, после чего суммарные моменты относительно центра масс КА имеют вид:
\begin{equation*}
	\begin{aligned}
		M_{X_0} &= F_{Z_0}R_Y + F_{Y_0}R_Z + (M_{da}-M_{ma})\sin\alpha, \\
		M_{Y_0} &= F_{X_0}R_Z + F_{Z_0}R_X + (M_{da}-M_{ma}), \\
		M_{Z_0} &= F_{Y_0}R_X + (M_{db}-M_{mb})\cos\alpha.
	\end{aligned}
\end{equation*}

Таким образом, результирующие реактивные моменты представляют собой сумму двух составляющих: 
\begin{itemize}
	\item моментов, вызванных смещением центра карданова подвеса относительно центра масс КА ($R_x, R_y, R_z$);
	\item остаточных моментов, возникающих из-за неполной компенсации приводами маховиков.
\end{itemize}


Отдельное внимание уделяется влиянию профиля разгона привода на спектр возмущающих воздействий. Трапецеидальный закон прост и время-оптимален при заданных ограничениях на $\alpha$ и $\omega$, но содержит скачки ускорения (рывок), что порождает широкополосные гармоники. Синусоидальный профиль обеспечивает непрерывность не только $\omega$ и $\alpha$, но и плавность изменений рывка. 
Для оценки оптимальности профиля используется критерий Рэлея, характеризующий отношение энергии рывка к энергии ускорения:

\begin{equation*}
	\label{eq:relay}
	\mathcal{R}[\epsilon] =
	\frac{\displaystyle \int_{0}^{T} \bigl(\dot{\epsilon}(t)\bigr)^{2}\,dt}
	{\displaystyle \int_{0}^{T} \epsilon^{2}(t)\,dt},
\end{equation*}
где $\epsilon(t)$ --- функция ускорения, $T$ --- длительность манёвра.

Минимальное значение функционала достигается для гармонического профиля ускорения, что можно показать разложением в ряд Фурье. 
При граничных условиях
\begin{equation*}
	\epsilon(0) = \epsilon(T) = 0, 
	\qquad 
	\int_{0}^{T} \epsilon(t)\,dt = 0,
\end{equation*}
первой допустимой собственной функцией является синус с удвоенной частотой, и оптимальный профиль имеет вид:
\begin{equation*}
	\epsilon(t) = \epsilon_0 \sin\!\left(\frac{2\pi t}{T}\right).
\end{equation*}


Таким образом, синусоидальный закон обеспечивает минимум функционала Рэлея, то есть наилучшее сглаживание ускорений и рывков. Его спектр ограничен одной гармоникой, что минимизирует возбуждение высокочастотных колебаний и снижает уровень микровибраций, передаваемых на основание космического аппарата.

Результаты, полученные в данной главе, легли в основу методики экспериментального измерения реактивных моментов и проектирования системы компенсации, рассмотренных в последующих разделах работы.



\underline{\textbf{Третья глава}} посвящена разработке и созданию испытательного стенда для измерения остаточных реактивных моментов оптико-механических систем.
Целью разработанной методики является обеспечение достоверного определения реактивных моментов, возникающих при работе приводов оптико-механических систем (ОМС), с учётом их динамического взаимодействия с корпусом. Методика основана на измерении углового ускорения инерционного звена, вызванного воздействием реактивного момента, и сравнении полученного отклика с откликом на эталонное воздействие известной величины.
Для описания динамики стенда используется модель колебательного звена второго порядка:
\begin{samepage}
	\begin{equation*}
		\label{eq:stadeq}
		J\cdot \frac{d^2\varphi}{dt^2}+b \cdot \frac{d\varphi}{dt}+ c \cdot \varphi = M(t)
	\end{equation*}
	
	где \(J\) --- момент инерции рамы вместе с установленной оптической системой, \(\varphi\) --- угол поворота рамы относительно равновесного положения, \(b\) --- коэффициент затухания, \(с\) --- коэффициент крутильной жёсткости, \(M(t)\) --- реактивный момент.
	\end{samepage}


При проектировании установки обеспечивалось выполнение условий, при которых собственная частота маятника смещена относительно диапазона частот возмущений, а система работает в послерезонансной области. Это позволяет использовать механическую часть стенда как фильтр низких частот, обеспечивающий подавление внешних вибраций и сохранение информативной составляющей реактивного момента.

Для исключения фазовых искажений в динамике звена обеспечен малый коэффициент затухания: $\xi\leq 0,1$. Моделирование показало, что при больших значениях декремента ($\xi \geq 0,1 $) наблюдается существенное снижение амплитуды и фазовый сдвиг измеренного сигнала относительно входного момента.

Экспериментальный стенд реализует принцип маятникового одноосного измерения измерения реактивного момента.
Подвижная часть подвешена на двух струнах и имеет одну степень свободы — вращение вокруг вертикальной оси (рис.~\cref{fig:yoiom}).Подвижная часть подвешена на двух струнах, обеспечивающих крутильную упругость и малое затухание. На ней размещён жёсткий каркас с посадочным местом для оптико-механической системы и балансировочными грузами. 

На подвесной раме установлен волоконно-оптический гироскоп, измеряющий угловую скорость $\omega(t)$. Для формирования эталонного воздействия используется тестовый маховик с известным моментом инерции $J_m$, создающий момент $M_m=J_m\alpha_m$, где $\alpha_m$ - ускорение задаваемое двигателю пультом управления.

Эти два элемента — гироскоп и маховик — образуют измерительно-калибровочную пару, обеспечивающую определение ускорений $\varepsilon_o$ (от исследуемого момента) и $\varepsilon_m$ (от эталонного).

\begin{figure}[!h] 
	\centerfloat{
		\includegraphics[scale=1]{yoim-cxem} 
	}
	\caption{Стенд измерения реактивного момента}
	\label{fig:yoiom} 
\end{figure}

ОМС устанавливается на кантователь, позволяющий проводить измерения по осям $OY$ и $OZ$.
При вращении подвижной части ОМС создаётся реактивный момент, вызывающий поворот рамы. Гироскоп регистрирует угловую скорость, по которой численным дифференцированием вычисляется угловое ускорение:

\begin{equation*}
	\label{eq:mean_acc}
	\varepsilon_{i}
	= \frac{\overline{\omega}_{i+1}-\overline{\omega}_{i}}{\Delta t},
	\qquad
	\Delta t = \frac{1}{f_s}.
\end{equation*}
где \(f_s\)"--- частота дискретизации.

Остаточный реактивный момент вычисляется сравнением ускорений при эталонном и исследуемом воздействиях:
\begin{equation*}
	M = M_m\frac{\varepsilon_o}{\varepsilon_m},
\end{equation*}
где \(M_m = J_m\alpha_m\)"--- известный тестовый момент маховика, \(\varepsilon_o\)"--- ускорение рамы, вызванное поворотом подвижной части оптико-механической системы, \(\varepsilon_m\)"--- ускорение вызванное тестовым моментом.


Разработанная методика прошла метрологическую аттестацию
в установленном порядке. Аттестация выполнена в АО «НПО Техномаш»
им. С.~А.~Афанасьева при поддержке Госкорпорации «Роскосмос».
По результатам выдано свидетельство №~030-500/2024-61
(РОСС~RU.0001.310066/2024), запись в государственном реестре
ФР.1.28.2024.49055. Методика измерений признана соответствующей
ГОСТ~Р~8.563–2009. Диапазон измеряемых моментов составляет
от $10^{-3}$ до $1\,\text{Н}\cdot\text{м}$ при относительной
неопределённости порядка $2\,\%$, что подтверждает пригодность стенда
для поверочных работ и исследовательских испытаний оптико-механических систем.

В данной главе разработана методика измерения остаточных реактивных моментов оптико-механических систем и создан испытательный стенд, реализующий предложенный принцип.
На основе модели колебательного звена второго порядка обоснованы требования к частотным характеристикам и демпфированию подвесной системы.
Спроектирован маятниковый стенд с волоконно-оптическим гироскопом и тестовым маховиком, обеспечивающий одноканальное измерение реактивных моментов в диапазоне от $10^{-3}$ до $1\,\text{Н}\cdot\text{м}$. Проведённая метрологическая аттестация подтвердила соответствие методики требованиям ГОСТ Р 8.563–2009 и показала относительную неопределённость не более 2 \%. Разработанный комплекс обеспечивает достоверную экспериментальную оценку реактивных моментов и может применяться для поверочных и исследовательских испытаний оптико-механических систем космического назначения.

\underline{\textbf{Четвертая главе}} посвящена экспериментальным исследованиям и проверке работоспособности разработанного стенда.

Для проверки достоверности методики выполнена серия измерений реактивного момента модуля направленного обзора.
В качестве опорного воздействия использовался момент $M_{\text{тест}}=\SI{0.005}{\newton\meter}$;
регистрировалась угловая скорость колебаний, затем выполнялось усреднение по нескольким измерениям и численное дифференцирование (рис.~\cref{fig:test-gyro-acc}).

\begin{figure}[h!]
	\centering
	\includegraphics[scale=0.35]{matlab/img/test-gyro-acc}
	\caption{Ускорение при тестовом воздействии}
	\label{fig:test-gyro-acc}
\end{figure}

Затем выполнены повороты ОМС вокруг осей $OZ$ и $OY$ (по 10 реализаций), обработка проводилась аналогично:
усреднение по ансамблю, дифференцирование
(рис.~\cref{fig:oz-gyro}).

\begin{figure}[!h]
	\begin{minipage}[b]{0.49\linewidth}\centering
		\includegraphics[width=0.8\linewidth]{matlab/img/oy-gyro-acc.png}\\[-2pt] a)
	\end{minipage}
	\hfill
	\begin{minipage}[b]{0.49\linewidth}\centering
		\includegraphics[width=0.8\linewidth]{matlab/img/oz-gyro-acc.png}\\[-2pt] б)
	\end{minipage}
	\caption{Ускорение рамы при поворот ОМС вокруг a) $OY$ б) $OZ$}
	\label{fig:oz-gyro}
\end{figure}

Остаточный момент ОМС определялся сравнением с эталоном:
\[
M_{\text{ОМС}} = M_{\text{тест}}\cdot\frac{\varepsilon_{\text{ОМС}}}{\varepsilon_{\text{тест}}}\!.
\]

По результатам обработки экспериментальных данных величина реактивного момента составила $M_y = \SI{0,076}{\newton\meter}$ при повороте вокруг оси $Y$ и $M_z = \SI{0,055}{\newton\meter}$ при повороте вокруг оси $Z$ (см. рис.~\cref{fig:omn-mom}).

\begin{figure}[!h]
	\begin{minipage}[b]{0.49\linewidth}\centering
		\includegraphics[width=0.8\linewidth]{matlab/img/oy-gyro-mom.png}\\[-2pt] a)
	\end{minipage}
	\hfill
	\begin{minipage}[b]{0.49\linewidth}\centering
		\includegraphics[width=0.8\linewidth]{matlab/img/oz-gyro-mom.png}\\[-2pt] б)
	\end{minipage}
	\caption{Реактивный момент при поворотах вокруг а) $OZ$ б) $OY$}
	\label{fig:omn-mom}
\end{figure}

Для усиления компенсации были подобраны и установлены дополнительные балансировочные кольца на маховики  по осям $OY$ и $OZ$ с моментами инерции, равными $J_{my}=\num{0.0015}\,\si{\kilo\gram\meter\squared}$ и $J_{mz}=\num{0.0011}\,\si{\kilo\gram\meter\squared}$ соответственно. После установки колец проведены повторные стендовые измерения с использованием синусоидального профиля разгона привода.


На рисунке~\cref{fig:sin-profile-omn} представлены результаты измерений после установки балансировочных колец и применения синусоидального профиля.
\begin{figure}[h!]
	% --- Первая строка ---
	\begin{minipage}[b]{0.49\linewidth}\centering
		\includegraphics[width=\linewidth]{matlab/img/oz-gyro-sin-vel} \\ а) угловая скорость (ось $OZ$)
	\end{minipage}
	\hfill
	\begin{minipage}[b]{0.49\linewidth}\centering
		\includegraphics[width=\linewidth]{matlab/img/oy-gyro-sin-vel} \\ б) Угловая скорость (ось $OY$)
	\end{minipage}
	
	\vspace{0.5em} % расстояние между строками
	
	% --- Вторая строка ---
	\begin{minipage}[b]{0.49\linewidth}\centering
		\includegraphics[width=\linewidth]{matlab/img/oz-gyro-sin-acc} \\ в) угловое ускорение (ось $OZ$)
	\end{minipage}
	\hfill
	\begin{minipage}[b]{0.49\linewidth}\centering
		\includegraphics[width=\linewidth]{matlab/img/oy-gyro-sin-acc} \\ г)  угловое ускорение (ось $OY$)
	\end{minipage}
	
	\vspace{0.5em} % расстояние между строками
	
	\begin{minipage}[b]{0.49\linewidth}\centering
		\includegraphics[width=\linewidth]{matlab/img/oz-gyro-sin-mom} \\ д) реактивный момент (ось $OZ$)
	\end{minipage}
	\hfill
	\begin{minipage}[b]{0.49\linewidth}\centering
		\includegraphics[width=\linewidth]{matlab/img/oy-gyro-sin-mom} \\ е)  реактивный момент (ось $OY$)
	\end{minipage}
	
	\caption{Измерения после установки балансировочных колец}
	\label{fig:sin-profile-omn}
\end{figure}

Сравнение полученных зависимостей с результатами первой серии испытаний показало, что после установки балансировочных колец величина реактивного момента снизилась, а форма кривых стала более симметричной и плавной. Применение синусоидального профиля управления дополнительно уменьшило амплитуду ускорений и сгладило переходные процессы, что привело к снижению возбуждения колебаний подвесной системы. Таким образом, доработанная конструкция ОМС в сочетании с оптимизированным законом управления обеспечивает более эффективную компенсацию реактивного момента и повышает точность стабилизации оптической системы.

Доработанный таким образом лётный образец ОМС был запущен на орбиту. Для оценки влияния остаточного реактивного момента в условиях полёта
проанализированы данные бортовых гироскопов КА.

 \begin{figure}[h!]
	\begin{minipage}[b][][b]{0.49\linewidth}\centering
		\includegraphics[width=1\linewidth]{matlab/img/sat_gyro_dataY.png} \\ a)
	\end{minipage}
	\hfill
	\begin{minipage}[b][][b]{0.49\linewidth}\centering
		\includegraphics[width=1\linewidth]{matlab/img/sat_gyro_dataZ.png} \\ б)
	\end{minipage}
	\caption{Скорость угловых колебаний спутника при повороте оси визирования по a) оси $OY$, б) оси $OZ$ }
	\label{fig:rotationYZ}
\end{figure}

В моменты максимальных разворотов оси визирования регистрировались угловые скорости
$\omega_y(t), \omega_z(t), \omega_x(t)$ (Рисунок~\cref{fig:rotationYZ}).
Интегрированием получены углы поворота:
\begin{equation*}
	\theta_y(t)=\int \omega_y\,dt,\quad
	\theta_z(t)=\int \omega_z\,dt,\quad
	\theta_x(t)=\int \omega_x\,dt.
\end{equation*}

Смещение изображения в фокальной плоскости определяется как
\begin{equation*}
	\label{eq:bias}
	\mathbf{r}(t) = 
	\begin{bmatrix}
		y(t) \\
		z(t) \\
	\end{bmatrix}
	= f \cdot
	\begin{bmatrix}
		\theta_{y}(t) \\
		\theta_{z}(t)
	\end{bmatrix}
\end{equation*}
где $f=\SI{900}{\milli\meter}$ — фокусное расстояние.
За время экспозиции $T_{\mathrm{exp}}$ линейное смещение составляет:
 \begin{equation*}
	\label{eq:biasL}
	L=\sqrt{(\Delta y)^2+(\Delta z)^2}
\end{equation*}
а угловое смещение — $\alpha=R\theta_x$, где $R=\SI{30}{\milli\meter}$ — радиус кадра.
Их совместное действие даёт результирующий сдвиг $L_{\mathrm{tot}}$.

 \begin{equation*}
	\label{eq:L_total}
	L_{\sum} = \sqrt{L^2 + (R\cdot \theta_x)^2 + 2Lr\theta_x\cos{\psi}}.
\end{equation*}

Расчёты показали смещения
$L_{OY}=\SI{2.13}{\micro\meter}$ и
$L_{OZ}=\SI{3.09}{\micro\meter}$,
что составляет $0.07$ и $0.10$ пикселя.

Деградация контрастно-частотной характеристики оценивается по следующей формуле:
\begin{equation*}
	\text{ЧКХ}(\nu)=\frac{\sin(\pi\nu L_{\mathrm{\sum}})}{\pi\nu L_{\mathrm{\sum}}},\qquad
	\nu_N=\frac{1}{2p},
\end{equation*}
где $p=\SI{30}{\micro\meter}$ — размер пикселя.

 На частоте Найквиста получены
\[
ЧКХ(\nu_N)_{OY}\approx 0.998,\qquad
ЧКХ(\nu_N)_{OZ}\approx 0.996.
\]

Таким образом, остаточный реактивный момент не оказывает критического
влияния на формирование изображения, что подтверждает эффективность
мер по компенсации. Система компенсации реактивного момента,
параметры которой были оптимизированы с использованием разработанного
измерительного стенда, сводит величину возмущающих воздействий к
уровню, не влияющему на качество изображения.


Разработанный стенд может использоваться не только для исследования модулей с дискретными поворотами, но и для настройки оптических систем с периодическими колебательными движениями зеркал. В качестве примера рассмотрен сканер, в котором компенсация реактивного момента осуществляется отдельным маховиком с собственным приводом. Для таких систем предъявляются повышенные требования: остаточный момент должен быть менее $\SI{0,005}{\newton\meter}$, что требует высокой точности согласования работы обоих приводов.

Испытания на стенде включали подбор момента инерции маховика и настройку фазового совпадения его движения с движением зеркала. Результаты (рис.~\ref{fig:scan-mom}) показали, что после настройки уровень остаточного реактивного момента значительно снизился и соответствовал требуемым нормам.

\begin{figure}[h!]
	\begin{minipage}[b]{0.49\linewidth}\centering
		\includegraphics[width=1\linewidth]{matlab/img/scanner_no_sinchron} \\ а)
	\end{minipage}
	\hfill
	\begin{minipage}[b]{0.49\linewidth}\centering
		\includegraphics[width=1\linewidth]{matlab/img/scanner_correct} \\ б)
	\end{minipage}
	\caption{Реактивный момент оптической системы: а) до настройки; б) после настройки}
	\label{fig:scan-mom}
\end{figure}

В результате проведённых экспериментальных исследований подтверждена работоспособность и достоверность разработанной методики измерения реактивных моментов.
Испытания показали, что установка балансировочных колец и применение синусоидального профиля разгона приводят к снижению амплитуды реактивных моментов и уменьшению вибрационных воздействий на конструкцию.
Анализ данных полёта подтвердил, что остаточные моменты не оказывают существенного влияния на качество изображений, а контрастно-частотная характеристика системы сохраняется на уровне не ниже 0,996 от номинального значения.
Разработанный стенд продемонстрировал универсальность и может применяться для настройки и испытаний различных типов оптико-механических систем.







\FloatBarrier
\pdfbookmark{Заключение}{conclusion}                                  % Закладка pdf
\section*{Заключение}
%% Согласно ГОСТ Р 7.0.11-2011:
%% 5.3.3 В заключении диссертации излагают итоги выполненного исследования, рекомендации, перспективы дальнейшей разработки темы.
%% 9.2.3 В заключении автореферата диссертации излагают итоги данного исследования, рекомендации и перспективы дальнейшей разработки темы.


Анализ существующих методов показал необходимость в создании специализированного стенда для прямого измерения реактивного момента, возникающего при работе оптико-механических систем космического аппарата. Установлено, что традиционные способы регистрации крутящего момента на валу двигателя не позволяют получить полное представление о суммарном воздействии на конструкцию и не учитывают динамических факторов, влияющих на качество изображения.

В работе разработаны математические модели формирования реактивных моментов, учитывающие смещение центра масс и особенности компенсации маховиками. Показано, что величина остаточного момента во многом определяется не только конструктивными параметрами, но и профилем разгона приводов. Оптимальным является синусоидальный закон движения, минимизирующий энергию рывка и снижающий уровень возбуждаемых колебаний.

Создан и аттестован испытательный стенд, реализующий методику измерения реактивных моментов по угловым колебаниям подвесной системы. Метрологическая аттестация подтвердила диапазон измерений от $10^{-3}$ до $1\,\text{Н}\cdot\text{м}$ при относительной погрешности не более 1~\%.

Проведённые стендовые испытания показали хорошее совпадение с расчётными данными. Установка балансировочных колец и использование синусоидального профиля привода позволили снизить уровень остаточных моментов и сгладить переходные процессы. Достоверность результатов подтверждена анализом данных бортовых гироскопов на лётных испытания.

Разработанная методика может применяться не только для модулей с дискретными поворотами, но и для систем с колебательным режимом работы, что подтверждено испытаниями оптико-механического сканера. С помощью стенда удалось настроить параметры компенсационного маховика и синхронизировать работу приводов, обеспечив снижение остаточного реактивного момента до требуемого уровня.




\pdfbookmark{Литература}{bibliography}                                % Закладка pdf

\section*{Публикации в изданиях, рекомендованных ВАК России}
\begin{enumerate}
\item  Стенд измерения остаточного реактивного момента оптико-механической системы / Белан~И.~М., Ларионов~Ю.~П., Ларионов~Д.~Ю.  // Оптический журнал. "--- 2023. "--- Т.~90, №~7. "--- С.~60--67.

\item Оценка реактивного некомпенсированного момента оптико-механической системы / Ларионов Д.~Ю., Белан И.~М. // Известия СПбГЭТУ «ЛЭТИ». – 2024. – Т.~17. – №~7. – С.~16--22.

\item Влияние реактивного момента на размытие изображения / Белан И.~М. //  Известия высших учебных заведений России. Радиоэлектроника (в печати)




\end{enumerate}

\ifdefmacro{\microtypesetup}{\microtypesetup{protrusion=false}}{} % не рекомендуется применять пакет микротипографики к автоматически генерируемому списку литературы
\urlstyle{rm}                               % ссылки URL обычным шрифтом
\ifnumequal{\value{bibliosel}}{0}{% Встроенная реализация с загрузкой файла через движок bibtex8
    \renewcommand{\bibname}{\large \bibtitleauthor}
    \nocite{*}
    \insertbiblioauthor           % Подключаем Bib-базы
    %\insertbiblioexternal   % !!! bibtex не умеет работать с несколькими библиографиями !!!
}{% Реализация пакетом biblatex через движок biber
    % Цитирования.
    %  * Порядок перечисления определяет порядок в библиографии (только внутри подраздела, если `\insertbiblioauthorgrouped`).
    %  * Если не соблюдать порядок "как для \printbibliography", нумерация в `\insertbiblioauthor` будет кривой.
    %  * Если цитировать каждый источник отдельной командой --- найти некоторые ошибки будет проще.
    %


    \ifnumgreater{\value{usefootcite}}{0}{
        \begin{refcontext}[labelprefix={}]
            \ifnum \value{bibgrouped}>0
                \insertbiblioauthorgrouped    % Вывод всех работ автора, сгруппированных по источникам
            \else
                \insertbiblioauthor      % Вывод всех работ автора
            \fi
        \end{refcontext}
    }{
        \ifnum \totvalue{citeexternal}>0
            \begin{refcontext}[labelprefix=A]
                \ifnum \value{bibgrouped}>0
                    \insertbiblioauthorgrouped    % Вывод всех работ автора, сгруппированных по источникам
                \else
                    \insertbiblioauthor      % Вывод всех работ автора
                \fi
            \end{refcontext}
        \else
            \ifnum \value{bibgrouped}>0
                \insertbiblioauthorgrouped    % Вывод всех работ автора, сгруппированных по источникам
            \else
                \insertbiblioauthor      % Вывод всех работ автора
            \fi
        \fi
        %  \insertbiblioauthorimportant  % Вывод наиболее значимых работ автора (определяется в файле characteristic во второй section)
        \begin{refcontext}[labelprefix={}]
            \insertbiblioexternal            % Вывод списка литературы, на которую ссылались в тексте автореферата
        \end{refcontext}
        % Невидимый библиографический список для подсчёта количества внешних публикаций
        % Используется, чтобы убрать приставку "А" у работ автора, если в автореферате нет
        % цитирований внешних источников.
        \printbibliography[heading=nobibheading, section=0, env=countexternal, keyword=biblioexternal, resetnumbers=true]%
    }
}
\ifdefmacro{\microtypesetup}{\microtypesetup{protrusion=true}}{}
\urlstyle{tt}                               % возвращаем установки шрифта ссылок URL
