\thispagestyle{empty}

\noindent%
\begin{flushright}
	\large{На правах рукописи}
\end{flushright}
\begin{tabularx}{\textwidth}{@{}lXr@{}}%
    
    \IfFileExists{images/logo.pdf}{\includegraphics[height=2.5cm]{logo}}{\rule[0pt]{0pt}{2.5cm}}  & &
    \ifnumequal{\value{showperssign}}{0}{%
        \rule[0pt]{0pt}{1.5cm}
    }{
        %\includegraphics[height=1.5cm]{personal-signature.png}
    }\\
\end{tabularx}

\vspace{0pt plus1fill} %число перед fill = кратность относительно некоторого расстояния fill, кусками которого заполнены пустые места
\begin{center}
\textbf {\large \thesisAuthor}
\end{center}

\vspace{0pt plus3fill} %число перед fill = кратность относительно некоторого расстояния fill, кусками которого заполнены пустые места
\begin{center}
\textbf {\Large %\MakeUppercase
\thesisTitle}

\vspace{0pt plus3fill} %число перед fill = кратность относительно некоторого расстояния fill, кусками которого заполнены пустые места
{\large Специальность \thesisSpecialtyNumber\ "---\par <<\thesisSpecialtyTitle>>}

\ifdefined\thesisSpecialtyTwoNumber
{\large Специальность \thesisSpecialtyTwoNumber\ "---\par <<\thesisSpecialtyTwoTitle>>}
\fi

\vspace{0pt plus1.5fill} %число перед fill = кратность относительно некоторого расстояния fill, кусками которого заполнены пустые места
\Large{Автореферат}\par
\large{диссертации на соискание учёной степени\par \thesisDegree}
\end{center}

\vspace{0pt plus4fill} %число перед fill = кратность относительно некоторого расстояния fill, кусками которого заполнены пустые места
{\centering\thesisCity~--- \thesisYear\par}

\newpage
% оборотная сторона обложки
\thispagestyle{empty}
\noindent  Работа выполнена в федеральном государственном автономном образова
тельном учреждении высшего образования «Санкт-Петербургский госу
дарственный электротехнический университет «ЛЭТИ» им. В.И. Ульяно
ва (Ленина)» на кафедре лазерных измерительных и навигационных си
стем.

\vspace{0.008\paperheight plus1fill}
\noindent%
\begin{tabularx}{\textwidth}{@{}lX@{}}
    \ifdefined\supervisorTwoFio
    Научные руководители:   & \supervisorRegalia\par
                              \ifdefined\supervisorDead
                              \framebox{\textbf{\supervisorFio}}
                              \else
                              \textbf{\supervisorFio}
                              \fi
                              \par
                              \vspace{0.013\paperheight}
                              \supervisorRegalia\par
                              \ifdefined\supervisorTwoDead
                              \framebox{\textbf{\supervisorTwoFio}}
                              \else
                              \textbf{\supervisorTwoFio}
                              \fi
                              \vspace{0.013\paperheight}\\
    \else
    Научный руководитель:   & \supervisorRegalia\par
                              \ifdefined\supervisorDead
                              \framebox{\textbf{\supervisorFio}}
                              \else
                              \textbf{\supervisorFio}
                              \fi
                              \vspace{0.013\paperheight}\\
    \fi
    Официальные оппоненты:  &
    \ifnumequal{\value{showopplead}}{0}{\vspace{13\onelineskip plus1fill}}{%
        \textbf{\opponentOneFio,}\par
        \opponentOneRegalia,\par
        \opponentOneJobPlace,\par
        \opponentOneJobPost\par
        \vspace{0.01\paperheight}
        \textbf{\opponentTwoFio,}\par
        \opponentTwoRegalia,\par
        \opponentTwoJobPlace,\par
        \opponentTwoJobPost
    \ifdefined\opponentThreeFio
        \par
        \vspace{0.01\paperheight}
        \textbf{\opponentThreeFio,}\par
        \opponentThreeRegalia,\par
        \opponentThreeJobPlace,\par
        \opponentThreeJobPost
    \fi
    }%
    \vspace{0.013\paperheight} \\
    \ifdefined\leadingOrganizationTitle
    Ведущая организация:    &
    \ifnumequal{\value{showopplead}}{0}{\vspace{6\onelineskip plus1fill}}{%
        \leadingOrganizationTitle
    }%
    \fi
\end{tabularx}
\vspace{0.008\paperheight plus1fill}

\noindent Защита состоится \defenseDate~на~заседании диссертационного совета \defenseCouncilNumber~при \defenseCouncilTitle~по адресу: \defenseCouncilAddress.

\vspace{0.008\paperheight plus1fill}
\noindent С диссертацией можно ознакомиться в библиотеке \synopsisLibrary.

\vspace{0.008\paperheight plus1fill}
\noindent Отзывы на автореферат в двух экземплярах, заверенные печатью учреждения, просьба направлять по адресу: \defenseCouncilAddress, ученому секретарю диссертационного совета~\defenseCouncilNumber.

\vspace{0.008\paperheight plus1fill}
\noindent{Автореферат разослан \synopsisDate.}

\noindent Телефон для справок: \defenseCouncilPhone.

\vspace{0.008\paperheight plus1fill}
\noindent%
\begin{tabularx}{\textwidth}{@{}%
>{\raggedright\arraybackslash}b{18em}@{}
>{\centering\arraybackslash}X
r
@{}}
    Ученый секретарь\par
    диссертационного совета\par
    \defenseCouncilNumber,\par
    \defenseSecretaryRegalia
    &
    \ifnumequal{\value{showsecrsign}}{0}{}{%
        \includegraphics[width=2cm]{secretary-signature.png}%
    }%
    &
    \defenseSecretaryFio
\end{tabularx}
