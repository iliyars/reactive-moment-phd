%% Согласно ГОСТ Р 7.0.11-2011:
%% 5.3.3 В заключении диссертации излагают итоги выполненного исследования, рекомендации, перспективы дальнейшей разработки темы.
%% 9.2.3 В заключении автореферата диссертации излагают итоги данного исследования, рекомендации и перспективы дальнейшей разработки темы.


Анализ существующих методов показал необходимость в создании специализированного стенда для прямого измерения реактивного момента, возникающего при работе оптико-механических систем космического аппарата. Установлено, что традиционные способы регистрации крутящего момента на валу двигателя не позволяют получить полное представление о суммарном воздействии на конструкцию и не учитывают динамических факторов, влияющих на качество изображения.

В работе разработаны математические модели формирования реактивных моментов, учитывающие смещение центра масс и особенности компенсации маховиками. Показано, что величина остаточного момента во многом определяется не только конструктивными параметрами, но и профилем разгона приводов. Оптимальным является синусоидальный закон движения, минимизирующий энергию рывка и снижающий уровень возбуждаемых колебаний.

Создан и аттестован испытательный стенд, реализующий методику измерения реактивных моментов по угловым колебаниям подвесной системы. Метрологическая аттестация подтвердила диапазон измерений от $10^{-3}$ до $1\,\text{Н}\cdot\text{м}$ при относительной погрешности не более 1~\%.

Проведённые стендовые испытания показали хорошее совпадение с расчётными данными. Установка балансировочных колец и использование синусоидального профиля привода позволили снизить уровень остаточных моментов и сгладить переходные процессы. Достоверность результатов подтверждена анализом данных бортовых гироскопов на лётных испытания.

Разработанная методика может применяться не только для модулей с дискретными поворотами, но и для систем с колебательным режимом работы, что подтверждено испытаниями оптико-механического сканера. С помощью стенда удалось настроить параметры компенсационного маховика и синхронизировать работу приводов, обеспечив снижение остаточного реактивного момента до требуемого уровня.


