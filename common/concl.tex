%% Согласно ГОСТ Р 7.0.11-2011:
%% 5.3.3 В заключении диссертации излагают итоги выполненного исследования, рекомендации, перспективы дальнейшей разработки темы.
%% 9.2.3 В заключении автореферата диссертации излагают итоги данного исследования, рекомендации и перспективы дальнейшей разработки темы.


В результате выполненных исследований достигнута цель работы — разработана и экспериментально подтверждена методика прямого измерения реактивных моментов, возникающих при работе оптико-механических систем космических аппаратов.

Анализ существующих подходов показал, что традиционные методы регистрации крутящего момента на валу двигателя не позволяют оценить суммарное воздействие на конструкцию и не учитывают динамических факторов, определяющих качество формируемого изображения.

Разработана математическая модель формирования реактивных моментов, учитывающая смещение центра масс, особенности конструкции карданового подвеса и компенсацию маховиками. Показано, что величина остаточного момента определяется как конструктивными параметрами, так и законом управления приводом. Обосновано применение синусоидального профиля разгона, обеспечивающего минимизацию энергии рывка и снижение уровня возбуждаемых колебаний.

Разработана и метрологически аттестована методика измерения реактивных моментов по угловым колебаниям подвесной системы, реализованная на созданном испытательном стенде. Аттестация подтвердила диапазон измеренийот $10^{-3}$ до $1\,\text{Н}\cdot\text{м}$ при относительной погрешности не более 2~\%.

Результаты стендовых испытаний показали хорошее совпадение с расчётными данными. Применение балансировочных колец и синусоидального профиля управления приводом позволило снизить уровень остаточных моментов и сгладить переходные процессы. Достоверность результатов подтверждена анализом данных бортовых гироскопов на лётных испытаниях.

Разработанная методика и созданный стенд могут использоваться для настройки и испытаний различных типов оптико-механических систем, включая устройства с дискретными и периодическими движениями. Это обеспечивает повышение динамической точности, снижение остаточного реактивного момента и улучшение качества изображений, формируемых космическими оптическими аппаратами.