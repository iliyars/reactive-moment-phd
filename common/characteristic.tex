
{\actuality} В последнее время всё более очевидной становится тенденция к повышению точности и чувствительности космических средств, предназначенных для наблюдения и получения информации о положении малоэнергетических целей. Одним из путей решения этой задачи является увеличение размеров оптических систем космического назначения.

Расширение эксплуатационных возможностей такой широкоформатной оптики предполагает, в свою очередь, введение в оптическую систему элементов, позволяющих изменять в пространстве положение визирной оси оптической аппаратуры. Эту задачу можно решить либо поворотом космического аппарата (КА) в пространстве, либо за счет изменения положения одного либо нескольких элементов оптической системы относительно КА. Другим вариантом получения эффекта перенацеливания оптической системы является разворот всей оптической системы относительно КА.


Понятно, что с точки зрения экономии энергии на борту КА наиболее рациональным решением будет перемещение в процессе перенацеливания минимальной массы (одного или нескольких элементов оптической системы). Но, независимо от принятого выбора конструктивного исполнения, при изменении положения перемещаемой массы относительно КА возникнут реактивные силы и моменты, воздействующие на КА, которые приведут к развороту КА вокруг его центра масс в направлении, противоположном направлению перемещения подвижной массы. Таким образом, в результате взаимного перемещения оптической системы (или её элементов) относительно КА на некоторый заданный угол и перемещения самого КА в пространстве ось визирования оптической системы займёт в пространстве некоторое положение, не совпадающее с заданными углами на перенацеливание.  Особенно сильно влияние реактивных моментов и сил в случае инфракрасных оптических систем космического назначения, имеющих значительные габариты массу. 

Как правило, на борту КА функционирует система стабилизации положения КА в пространстве. В результате работы этой системы через некоторое время (значительно большее времени перенацеливания) КА вернётся в положение, которое он занимал до начала процесса перенацеливания. Только тогда ось визирования оптической системы постепенно займёт требуемое положение. Кроме того, на КА закреплено большое количество устройств (антенны, солнечные панели и т.д.), имеющих достаточно низкую частоту собственных колебаний. Если перенацеливание происходит за короткое время, то реактивное воздействие на КА близко по своей природе к ударному воздействию, что может привести к возникновению медленно затухающих собственных колебаний этих устройств относительно КА. Это обстоятельство приводит к возникновению дополнительных гармонических воздействий на КА, что затрудняет работу системы стабилизации КА и приводит к ещё большему затягиванию процесса перенацеливания оси визирования оптической системы.
Следует отметить, что данная тематика в настоящее время разработана недостаточно подробно и имеющийся список научной литературы по данному вопросу весьма скуден.

В связи с этим, актуально проведение подробного исследования результатов влияния реактивных моментов на КА, возникающих в процессе перенацеливания визирной оси оптической системы, входящей в состав КА.


% {\progress}
% Этот раздел должен быть отдельным структурным элементом по
% ГОСТ, но он, как правило, включается в описание актуальности
% темы. Нужен он отдельным структурынм элемементом или нет ---
% смотрите другие диссертации вашего совета, скорее всего не нужен.

{\aim}  является разработка методов оценки и снижения остаточного реактивного момента оптико-механической системы, обеспечивающих повышение динамической совместимости с системой ориентации и сохранение качества формируемого изображения.

Для~достижения поставленной цели необходимо было решить следующие {\tasks}:
\begin{enumerate}[beginpenalty=10000] % https://tex.stackexchange.com/a/476052/104425
  \item Анализ существующих методов оценки и компенсации реактивных моментов в подвижных системах летательных объектов;
  \item Построение математической модели формирования остаточного реактивного момента в оптико-механической системе;
  \item Исследование влияния остаточного реактивного момента на динамику космического аппарата и параметры качества изображения;
  \item Разработка методики экспериментальной оценки остаточных реактивных моментов в наземных условиях;
  \item Создание экспериментальной установки для воспроизведения и измерения реактивных воздействий;
  \item Анализ влияния законов управления приводами на возбуждение колебаний конструкции космического аппарата и характеристики переходного процесса.
\end{enumerate}


{\novelty}
\begin{enumerate}[beginpenalty=10000] % https://tex.stackexchange.com/a/476052/104425
  \item Впервые проведены расчёты пространственных реактивных воздействий на КА при проектировании нескольких вариантов крупногабаритных оптических систем космического назначения с возможностью вращения визирной оси, которые позволили оптимизировать конструкцию аппаратуры и алгоритм управления.
  \item Создана и апробирована экспериментальная установка для воспроизведения и измерения реактивных воздействий, позволяющая регистрировать моменты в диапазоне $1 \cdot 10^{-3}$ до $1\,\text{Н}\cdot\text{м}$ с относительной погрешностью не более 1,5 \%
\end{enumerate}

{\influence}:

\begin{enumerate}[beginpenalty=10000] % https://tex.stackexchange.com/a/476052/104425
	\item Разработана математическая модель формирования остаточного реактивного момента в оптико-механической системе, расхождение расчёта с экспериментальными данными составляет 10~\%, что позволяет использовать её при предварительном проектировании конструкции оптико-механической системы и компенсирующих элементов.
	\item Разработана и аттестована методика экспериментальной оценки реактивных моментов в наземных условиях, обеспечивающая регистрацию моментов в диапазоне $1 \cdot 10^{-3}$ до $1\,\text{Н}\cdot\text{м}$ с относительной погрешностью $1,5 ~\%$.
	\item На разработанной экспериментальной установке для одной из исследованных оптико-механических систем обеспечено снижение остаточного реактивного момента до уровня $\SI{0,005}{\newton\meter}$, путём эмпирического подбора момента инерции и закона разгона компенсирующего маховика.
\end{enumerate}


{\methods} Решение поставленных задач базируется на использовании основных положений классической механики, динамики твёрдого тела, методов анализа и синтеза систем управления, метрологии качества изображения в оптико-электронных системах. В работе применялись методы математического анализа и линейной алгебры, спектральный анализ и численное интегрирование, математическое моделирование, экспериментальные исследования динамических процессов, а так же современные вычислительные средства для обработки результатов.


{\defpositions}
\begin{enumerate}[beginpenalty=10000] % https://tex.stackexchange.com/a/476052/104425
  \item \dubios[математическая модель не может быть положением]{Математическая модель оптико-механической системы обеспечивает возможность определения величины остаточного реактивного момента.}
  \item Остаточный реактивный момент является источником колебаний космического аппарата, которые увеличивают нагрузку на систему ориентации и могут приводить к снижению качества изображения.
  \item Использование результатов наземных измерений реактивного момента позволяет прогнозировать динамическое воздействие на космическая аппарат
  \item Применение разработанного измерительного стенда позволяет повысить точность оценки остаточного реактивного момента оптико-механической системы.
  \item Закон управления приводами определяет спектральные характеристики возмущений и время успокоение остаточных колебаний 
\end{enumerate}

{\realisation} Работа проведена в рамках выполнения государственного контракта №099-К260/21/23 тема \flqq Разработка технологии изготовления и испытаний прецизионных зеркальных сканирующих оптико-механических систем для оптико-электронной аппаратуры космического базирования на основе многоядерных крупноформатных фотоприемных устройств\frqq  в филиале АО \flqq Корпорация \glqq Комета \grqq -- \glqq НПЦ ОЭКН\grqq \frqq. Тема диссертационной работы непосредственно связана с работами по созданию прецизионных оптических систем, проводимыми в филиале, и направлена на повышение их эффективности и надёжности. Основные теоретически и практические результаты были внедрены в протоколы испытания оптических систем, разрабатываемых в корпорации.

{\probation}
Основные результаты работы докладывались~на:
перечисление основных конференций, симпозиумов и~т.\:п.


\ifnumequal{\value{bibliosel}}{0}
{%%% Встроенная реализация с загрузкой файла через движок bibtex8. (При желании, внутри можно использовать обычные ссылки, наподобие `\cite{vakbib1,vakbib2}`).
    {\publications} Основные результаты по теме диссертации изложены
    в~XX~печатных изданиях,
    X из которых изданы в журналах, рекомендованных ВАК,
    X "--- в тезисах докладов.
}%
{%%% Реализация пакетом biblatex через движок biber
    \begin{refsection}[bl-author, bl-registered]
        % Это refsection=1.
        % Процитированные здесь работы:
        %  * подсчитываются, для автоматического составления фразы "Основные результаты ..."
        %  * попадают в авторскую библиографию, при usefootcite==0 и стиле `\insertbiblioauthor` или `\insertbiblioauthorgrouped`
        %  * нумеруются там в зависимости от порядка команд `\printbibliography` в этом разделе.
        %  * при использовании `\insertbiblioauthorgrouped`, порядок команд `\printbibliography` в нём должен быть тем же (см. biblio/biblatex.tex)
        %
        % Невидимый библиографический список для подсчёта количества публикаций:
        \phantom{\printbibliography[heading=nobibheading, section=1, env=countauthorvak,          keyword=biblioauthorvak]%
        \printbibliography[heading=nobibheading, section=1, env=countauthorwos,          keyword=biblioauthorwos]%
        \printbibliography[heading=nobibheading, section=1, env=countauthorscopus,       keyword=biblioauthorscopus]%
        \printbibliography[heading=nobibheading, section=1, env=countauthorconf,         keyword=biblioauthorconf]%
        \printbibliography[heading=nobibheading, section=1, env=countauthorother,        keyword=biblioauthorother]%
        \printbibliography[heading=nobibheading, section=1, env=countregistered,         keyword=biblioregistered]%
        \printbibliography[heading=nobibheading, section=1, env=countauthorpatent,       keyword=biblioauthorpatent]%
        \printbibliography[heading=nobibheading, section=1, env=countauthorprogram,      keyword=biblioauthorprogram]%
        \printbibliography[heading=nobibheading, section=1, env=countauthor,             keyword=biblioauthor]%
        \printbibliography[heading=nobibheading, section=1, env=countauthorvakscopuswos, filter=vakscopuswos]%
        \printbibliography[heading=nobibheading, section=1, env=countauthorscopuswos,    filter=scopuswos]}%
        %
        \nocite{*}%
        %
        {\publications} Основные результаты по теме диссертации изложены в~\arabic{citeauthor}~печатных изданиях,
        \arabic{citeauthorvak} из которых изданы в журналах, рекомендованных ВАК%
        \ifnum \value{citeauthorscopuswos}>0%
            , \arabic{citeauthorscopuswos} "--- в~периодических научных журналах, индексируемых Web of~Science и Scopus%
        \fi%
        \ifnum \value{citeauthorconf}>0%
            , \arabic{citeauthorconf} "--- в~тезисах докладов.
        \else%
            .
        \fi%
        \ifnum \value{citeregistered}=1%
            \ifnum \value{citeauthorpatent}=1%
                Зарегистрирован \arabic{citeauthorpatent} патент.
            \fi%
            \ifnum \value{citeauthorprogram}=1%
                Зарегистрирована \arabic{citeauthorprogram} программа для ЭВМ.
            \fi%
        \fi%
        \ifnum \value{citeregistered}>1%
            Зарегистрированы\ %
            \ifnum \value{citeauthorpatent}>0%
            \formbytotal{citeauthorpatent}{патент}{}{а}{}%
            \ifnum \value{citeauthorprogram}=0 . \else \ и~\fi%
            \fi%
            \ifnum \value{citeauthorprogram}>0%
            \formbytotal{citeauthorprogram}{программ}{а}{ы}{} для ЭВМ.
            \fi%
        \fi%
        % К публикациям, в которых излагаются основные научные результаты диссертации на соискание учёной
        % степени, в рецензируемых изданиях приравниваются патенты на изобретения, патенты (свидетельства) на
        % полезную модель, патенты на промышленный образец, патенты на селекционные достижения, свидетельства
        % на программу для электронных вычислительных машин, базу данных, топологию интегральных микросхем,
        % зарегистрированные в установленном порядке.(в ред. Постановления Правительства РФ от 21.04.2016 N 335)
    \end{refsection}%
    \begin{refsection}[bl-author, bl-registered]
        % Это refsection=2.
        % Процитированные здесь работы:
        %  * попадают в авторскую библиографию, при usefootcite==0 и стиле `\insertbiblioauthorimportant`.
        %  * ни на что не влияют в противном случае
        \nocite{vakbib2}%vak
        \nocite{patbib1}%patent
        \nocite{progbib1}%program
        \nocite{bib1}%other
        \nocite{confbib1}%conf
    \end{refsection}%
        %
        % Всё, что вне этих двух refsection, это refsection=0,
        %  * для диссертации - это нормальные ссылки, попадающие в обычную библиографию
        %  * для автореферата:
        %     * при usefootcite==0, ссылка корректно сработает только для источника из `external.bib`. Для своих работ --- напечатает "[0]" (и даже Warning не вылезет).
        %     * при usefootcite==1, ссылка сработает нормально. В авторской библиографии будут только процитированные в refsection=0 работы.
}

При использовании пакета \verb!biblatex! будут подсчитаны все работы, добавленные
в файл \verb!biblio/author.bib!. Для правильного подсчёта работ в~различных
системах цитирования требуется использовать поля:
\begin{itemize}
        \item \texttt{authorvak} если публикация индексирована ВАК,
        \item \texttt{authorscopus} если публикация индексирована Scopus,
        \item \texttt{authorwos} если публикация индексирована Web of Science,
        \item \texttt{authorconf} для докладов конференций,
        \item \texttt{authorpatent} для патентов,
        \item \texttt{authorprogram} для зарегистрированных программ для ЭВМ,
        \item \texttt{authorother} для других публикаций.
\end{itemize}
Для подсчёта используются счётчики:
\begin{itemize}
        \item \texttt{citeauthorvak} для работ, индексируемых ВАК,
        \item \texttt{citeauthorscopus} для работ, индексируемых Scopus,
        \item \texttt{citeauthorwos} для работ, индексируемых Web of Science,
        \item \texttt{citeauthorvakscopuswos} для работ, индексируемых одной из трёх баз,
        \item \texttt{citeauthorscopuswos} для работ, индексируемых Scopus или Web of~Science,
        \item \texttt{citeauthorconf} для докладов на конференциях,
        \item \texttt{citeauthorother} для остальных работ,
        \item \texttt{citeauthorpatent} для патентов,
        \item \texttt{citeauthorprogram} для зарегистрированных программ для ЭВМ,
        \item \texttt{citeauthor} для суммарного количества работ.
\end{itemize}
% Счётчик \texttt{citeexternal} используется для подсчёта процитированных публикаций;
% \texttt{citeregistered} "--- для подсчёта суммарного количества патентов и программ для ЭВМ.

Для добавления в список публикаций автора работ, которые не были процитированы в
автореферате, требуется их~перечислить с использованием команды \verb!\nocite! в
\verb!Synopsis/content.tex!.
