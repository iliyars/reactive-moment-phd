
{\actuality} В последнее время всё более очевидной становится тенденция к повышению точности и чувствительности космических средств, предназначенных для наблюдения и получения информации о положении наблюдаемых объектах. Одним из путей решения этой задачи является увеличение размеров оптических систем космического назначения.

Расширение эксплуатационных возможностей такой широкоформатной оптики предполагает, в свою очередь, введение в оптическую систему элементов, позволяющих изменять в пространстве положение визирной оси оптической аппаратуры. Эту задачу можно решить либо поворотом космического аппарата (КА) в пространстве, либо за счет изменения положения одного либо нескольких элементов оптической системы относительно КА. 


Понятно, что с точки зрения экономии энергии на борту КА наиболее рациональным решением будет перемещение минимальной массы (одного или нескольких элементов оптической системы). Но, независимо от принятого выбора конструктивного исполнения, при изменении положения перемещаемой массы относительно КА возникнут реактивные силы и моменты, воздействующие на КА, которые приведут к развороту КА вокруг его центра масс в направлении, противоположном направлению перемещения подвижной массы. Таким образом, в результате взаимного перемещения оптической системы (или её элементов) относительно КА на некоторый заданный угол и перемещения самого КА в пространстве ось визирования оптической системы займёт в пространстве некоторое положение, не совпадающее с заданными углами на перенацеливание.  Особенно сильно влияние реактивных моментов в случае применения инфракрасных оптических систем космического назначения, имеющих значительные габариты и массу. 

Как правило, на борту КА функционирует система стабилизации положения КА в пространстве. В результате работы этой системы через некоторое время (значительно большее времени перемещения оптической системы) КА вернётся в положение, которое он занимал до начала процесса поворота оптической системы. Только тогда ось визирования постепенно займёт требуемое положение. Кроме того, на КА закреплено большое количество устройств (антенны, солнечные панели и т.д.), имеющих достаточно низкую частоту собственных колебаний. Если поворот происходит за короткое время, то реактивное воздействие на КА близко по своей природе к ударному, что может привести к возникновению медленно затухающих собственных колебаний этих устройств относительно КА. Это обстоятельство приводит к возникновению дополнительных гармонических воздействий на КА, что затрудняет работу системы стабилизации КА и приводит к ещё большему затягиванию процесса стабилизации оси визирования оптической системы.
Следует отметить, что данная тематика в настоящее время исследована недостаточно подробно и объём доступной научной литературы по данному вопросу ограничен.

В связи с этим, актуально проведение подробного исследования результатов влияния реактивных моментов, возникающих в процессе поворота визирной оси оптической системы, на стабилизацию КА в пространстве и качество формируемого изображения.


% {\progress}
% Этот раздел должен быть отдельным структурным элементом по
% ГОСТ, но он, как правило, включается в описание актуальности
% темы. Нужен он отдельным структурынм элемементом или нет ---
% смотрите другие диссертации вашего совета, скорее всего не нужен.

{\aim}  является разработка методов оценки и снижения остаточного реактивного момента, направленных на повышение эффективности работы системы стабилизации положения КА.

Для~достижения поставленной цели необходимо решить следующие {\tasks}:
\begin{enumerate}[beginpenalty=10000] % https://tex.stackexchange.com/a/476052/104425
  \item Анализ существующих методов оценки и компенсации реактивных моментов в подвижных системах космических аппаратов;
  \item Построение математической модели формирования остаточного реактивного момента в оптико-механической системе;
  \item Исследование влияния остаточного реактивного момента на динамику космического аппарата и параметры качества изображения;
  \item Разработка методики экспериментальной оценки остаточных реактивных моментов в условиях наземных испытаниях;
  \item Создание экспериментальной установки для воспроизведения и измерения реактивных воздействий;
  \item Анализ качества изображения после компенсации на лётном образце оптической системы.
  
\end{enumerate}


{\novelty}
\begin{enumerate}[beginpenalty=10000] % https://tex.stackexchange.com/a/476052/104425
  \item Впервые проведены расчёты пространственных реактивных воздействий на КА при проектировании нескольких вариантов крупногабаритных оптических систем космического назначения с возможностью вращения визирной оси, которые позволили оптимизировать конструкцию аппаратуры и алгоритм управления.
  \item Создана и апробирована методика оценки реактивных моментов в наземных условиях.
  \item Создана и апробирована экспериментальная установка для воспроизведения и измерения реактивных воздействий, позволяющая регистрировать моменты в диапазоне от $2 \cdot 10^{-3}$ до $1\,\text{Н}\cdot\text{м}$ с относительной погрешностью не более 2 \%
\end{enumerate}

{\influence}:

\begin{enumerate}[beginpenalty=10000] % https://tex.stackexchange.com/a/476052/104425
	\item Математическая модель остаточного реактивного момента в оптико-механической системе характеризуется сходимостью с экспериментальными данными не менее 10~\%, что позволяет использовать ее при предварительном  проектировании конструкции оптико-механической системы и компенсирующих элементов.
	\item Методика оценки реактивных моментов в наземных условиях прошла апробацию на созданном в рамках исследований экспериментальном стенде и обеспечила регистрацию моментов в диапазоне от $1 \cdot 10^{-3}$ до $\SI{1}{\newton\meter}$ с относительной погрешностью $2 ~\%$.
\end{enumerate}


{\methods} Решение поставленных задач базируется на использовании основных положений классической механики, динамики твёрдого тела, методов анализа и синтеза систем управления, метрологии качества изображения в оптико-электронных системах. В работе применялись методы математического анализа и линейной алгебры, спектральный анализ и численное интегрирование, математическое моделирование, экспериментальные исследования динамических процессов, а так же современные вычислительные средства для обработки результатов.


{\defpositions}
\begin{enumerate}[beginpenalty=10000] % https://tex.stackexchange.com/a/476052/104425
  \item Математическая модель, описывающая динамику подвижного объекта в составе КА, позволяет оценить величины возникающих реактивных моментов.
  \item Методика, предполагающая измерение в наземных условиях реактивных моментов, возбуждаемых подвижным объектом в составе КА, позволяет сформулировать требования к системе управления и стабилизации движения такого объекта, что обеспечит компенсацию дестабилизирующего воздействия на положение КА в условиях космоса.
  \item Компенсация реактивного момента, возникающего при перенацеливании оптических систем, обеспечит требуемое качество снимков.
\end{enumerate}

{\realisation} Работа проведена в рамках выполнения государственного контракта №099-К260/21/23 тема \flqq Разработка технологии изготовления и испытаний прецизионных зеркальных сканирующих оптико-механических систем для оптико-электронной аппаратуры космического базирования на основе многоядерных крупноформатных фотоприемных устройств\frqq  в филиале АО \flqq Корпорация \glqq Комета \grqq -- \glqq НПЦ ОЭКН\grqq \frqq. Тема диссертационной работы непосредственно связана с работами по созданию прецизионных оптических систем, проводимыми в филиале, и направлена на повышение их эффективности и надёжности. Основные теоретические и практические результаты были внедрены в протоколы испытания оптических систем, разрабатываемых в корпорации.

{\probation}
Основные результаты работы докладывались и обсуждались на следующих конференциях:
\begin{enumerate}
	\item XV Международная конференция «Прикладная оптика–2022» 2022 Санкт-Петербург, Россия.
	\item XХV Всероссийская научно-техническая конференция молодых учёных «Навигация и управление движением» 2023 Санкт-Петербург, Россия.
	\item VI Научно-техническая конференция молодых ученых и специалистов «Будущее предприятия – в творчестве молодых» 2024 г. Санкт-Петербург. Россия.
\end{enumerate}

{\publications} Основные результаты по теме диссертации изложены
в~6~печатных изданиях,
3 из которых изданы в журналах, рекомендованных ВАК,
3 "--- в тезисах докладов.

\ifnumequal{\value{bibliosel}}{0}
{%%% Реализация пакетом biblatex через движок biber
    \begin{refsection}[bl-author, bl-registered]
        % Это refsection=1.
        % Процитированные здесь работы:
        %  * подсчитываются, для автоматического составления фразы "Основные результаты ..."
        %  * попадают в авторскую библиографию, при usefootcite==0 и стиле `\insertbiblioauthor` или `\insertbiblioauthorgrouped`
        %  * нумеруются там в зависимости от порядка команд `\printbibliography` в этом разделе.
        %  * при использовании `\insertbiblioauthorgrouped`, порядок команд `\printbibliography` в нём должен быть тем же (см. biblio/biblatex.tex)
        %
        % Невидимый библиографический список для подсчёта количества публикаций:
        \phantom{\printbibliography[heading=nobibheading, section=1, env=countauthorvak,          keyword=biblioauthorvak]%
        \printbibliography[heading=nobibheading, section=1, env=countauthorwos,          keyword=biblioauthorwos]%
        \printbibliography[heading=nobibheading, section=1, env=countauthorscopus,       keyword=biblioauthorscopus]%
        \printbibliography[heading=nobibheading, section=1, env=countauthorconf,         keyword=biblioauthorconf]%
        \printbibliography[heading=nobibheading, section=1, env=countauthorother,        keyword=biblioauthorother]%
        \printbibliography[heading=nobibheading, section=1, env=countregistered,         keyword=biblioregistered]%
        \printbibliography[heading=nobibheading, section=1, env=countauthorpatent,       keyword=biblioauthorpatent]%
        \printbibliography[heading=nobibheading, section=1, env=countauthorprogram,      keyword=biblioauthorprogram]%
        \printbibliography[heading=nobibheading, section=1, env=countauthor,             keyword=biblioauthor]%
        \printbibliography[heading=nobibheading, section=1, env=countauthorvakscopuswos, filter=vakscopuswos]%
        \printbibliography[heading=nobibheading, section=1, env=countauthorscopuswos,    filter=scopuswos]}%
        %
        \nocite{*}%
        %
        {\publications} Основные результаты по теме диссертации изложены в~\arabic{citeauthor}~печатных изданиях,
        \arabic{citeauthorvak} из которых изданы в журналах, рекомендованных ВАК%
        \ifnum \value{citeauthorscopuswos}>0%
            , \arabic{citeauthorscopuswos} "--- в~периодических научных журналах, индексируемых Web of~Science и Scopus%
        \fi%
        \ifnum \value{citeauthorconf}>0%
            , \arabic{citeauthorconf} "--- в~тезисах докладов.
        \else%
            .
        \fi%
        \ifnum \value{citeregistered}=1%
            \ifnum \value{citeauthorpatent}=1%
                Зарегистрирован \arabic{citeauthorpatent} патент.
            \fi%
            \ifnum \value{citeauthorprogram}=1%
                Зарегистрирована \arabic{citeauthorprogram} программа для ЭВМ.
            \fi%
        \fi%
        \ifnum \value{citeregistered}>1%
            Зарегистрированы\ %
            \ifnum \value{citeauthorpatent}>0%
            \formbytotal{citeauthorpatent}{патент}{}{а}{}%
            \ifnum \value{citeauthorprogram}=0 . \else \ и~\fi%
            \fi%
            \ifnum \value{citeauthorprogram}>0%
            \formbytotal{citeauthorprogram}{программ}{а}{ы}{} для ЭВМ.
            \fi%
        \fi%
        % К публикациям, в которых излагаются основные научные результаты диссертации на соискание учёной
        % степени, в рецензируемых изданиях приравниваются патенты на изобретения, патенты (свидетельства) на
        % полезную модель, патенты на промышленный образец, патенты на селекционные достижения, свидетельства
        % на программу для электронных вычислительных машин, базу данных, топологию интегральных микросхем,
        % зарегистрированные в установленном порядке.(в ред. Постановления Правительства РФ от 21.04.2016 N 335)
    \end{refsection}%
    \begin{refsection}[bl-author, bl-registered]
        % Это refsection=2.
        % Процитированные здесь работы:
        %  * попадают в авторскую библиографию, при usefootcite==0 и стиле `\insertbiblioauthorimportant`.
        %  * ни на что не влияют в противном случае
        \nocite{vakbib2}%vak
        \nocite{patbib1}%patent
        \nocite{progbib1}%program
        \nocite{bib1}%other
        \nocite{confbib1}%conf
    \end{refsection}%
        %
        % Всё, что вне этих двух refsection, это refsection=0,
        %  * для диссертации - это нормальные ссылки, попадающие в обычную библиографию
        %  * для автореферата:
        %     * при usefootcite==0, ссылка корректно сработает только для источника из `external.bib`. Для своих работ --- напечатает "[0]" (и даже Warning не вылезет).
        %     * при usefootcite==1, ссылка сработает нормально. В авторской библиографии будут только процитированные в refsection=0 работы.
}


