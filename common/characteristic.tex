
{\actuality} На борту космических аппаратов различного назначения устанавливаются приборы и устройства, содержащие вращающиеся элементы с электроприводом. К таким устройствам относятся поворотные панели солнечных батарей, сканирующие зеркала и оптико-электронные системы дистанционного зондирования Земли, обеспечивающие наведение оптической оси в заданное направление и непрерывное сканирование исследуемого пространства во время движения аппарата по орбите.

В соответствии с третьим законом Ньютона момент, развиваемый электродвигателем, прикладывается не только к ротору, но и к статору (реактивный момент), который через систему крепления передаётся на корпус космического аппарата. В результате корпус вращается в направлении, противоположном вращению ротора, с угловым ускорением, величина которого обратно пропорциональна моменту инерции конструкции. В динамических режимах разгона и торможения реактивный момент может существенно превышать компенсирующие возможности системы ориентации, вызывая недопустимые отклонения пространственного положения аппарата.

Особенно чувствительными к воздействию реактивных моментов оказываются крупногабаритные оптико-электронные системы, формирующие изображение. Даже незначительные колебания визирной оси приводят к снижению пространственного разрешения и размытию изображения.

Таким образом, исследование влияния реактивного момента, возникающего при работе электроприводов визирных систем, на качество формируемого изображения и разработка методов его измерения и компенсации являются актуальной научно-технической задачей, имеющей важное значение для повышения точности и надёжности функционирования космических оптико-электронных систем. 


% {\progress}
% Этот раздел должен быть отдельным структурным элементом по
% ГОСТ, но он, как правило, включается в описание актуальности
% темы. Нужен он отдельным структурынм элемементом или нет ---
% смотрите другие диссертации вашего совета, скорее всего не нужен.

{\aim}  является разработка методов оценки и снижения остаточного реактивного момента, направленных на повышение эффективности работы системы стабилизации положения КА.

Для~достижения поставленной цели необходимо решить следующие {\tasks}:
\begin{enumerate}[beginpenalty=10000] % https://tex.stackexchange.com/a/476052/104425
  \item Анализ существующих методов оценки и компенсации реактивных моментов в подвижных системах космических аппаратов;
  \item Построение математической модели формирования остаточного реактивного момента в оптико-механической системе;
  \item Разработка профиля разгона исполнительного механизма, обеспечивающего снижение амплитуды реактивных моментов в переходных режимах;
  \item Разработка методики экспериментальной оценки остаточных реактивных моментов в условиях наземных испытаниях;
  \item Создание экспериментальной установки для воспроизведения и измерения реактивных воздействий;
  \item Анализ качества изображения после компенсации на лётном образце оптической системы.
  
\end{enumerate}


{\novelty}
\begin{enumerate}[beginpenalty=10000] % https://tex.stackexchange.com/a/476052/104425
  \item Впервые проведены расчёты пространственных реактивных воздействий на КА при проектировании нескольких вариантов крупногабаритных оптических систем космического назначения с возможностью вращения визирной оси, которые позволили оптимизировать конструкцию аппаратуры и алгоритм управления.
  \item Создана и апробирована методика оценки реактивных моментов в наземных условиях.
  \item Создана и апробирована экспериментальная установка для воспроизведения и измерения реактивных воздействий, позволяющая регистрировать моменты в диапазоне от $10^{-3}$ до $1\,\text{Н}\cdot\text{м}$ с относительной погрешностью не более 2 \%
\end{enumerate}

{\influence}:

\begin{enumerate}[beginpenalty=10000] % https://tex.stackexchange.com/a/476052/104425
	\item Математическая модель остаточного реактивного момента в оптико-механической системе характеризуется сходимостью с экспериментальными данными не менее 10~\%, что позволяет использовать ее при предварительном  проектировании конструкции оптико-механической системы и компенсирующих элементов.
	\item Методика оценки реактивных моментов в наземных условиях прошла апробацию на созданном в рамках исследований экспериментальном стенде и обеспечила регистрацию моментов в диапазоне от $1 \cdot 10^{-3}$ до $\SI{1}{\newton\meter}$ с относительной погрешностью $2 ~\%$.
	\item Разработанная методика может быть использована для настройки и калибровки систем управления вращением различных оптических систем, обеспечивая повышение их точности и стабильности работы.
\end{enumerate}


{\methods} Решение поставленных задач базируется на использовании основных положений классической механики, динамики твёрдого тела, методов анализа и синтеза систем управления, метрологии качества изображения в оптико-электронных системах. В работе применялись методы математического анализа и линейной алгебры, спектральный анализ и численное интегрирование, математическое моделирование, экспериментальные исследования динамических процессов, а так же современные вычислительные средства для обработки результатов.


{\defpositions}
\begin{enumerate}[beginpenalty=10000] % https://tex.stackexchange.com/a/476052/104425
  \item Математическая модель динамики оптико-механической системы космического аппарата позволяет количественно оценить реактивные моменты и обосновывает параметры методики их экспериментального измерения.
  \item Методика, предполагающая измерение в наземных условиях реактивных моментов, возбуждаемых подвижным объектом в составе КА, позволяет сформулировать требования к системе управления и стабилизации движения такого объекта, что обеспечит компенсацию дестабилизирующего воздействия на положение КА в условиях космоса.
  \item Компенсация реактивного момента, возникающего при динамических режимах наведения оптических систем, позволяет сохранять частотно-контрастную характеристику оптической системы на уровне, обеспечивающем требуемое качество снимков.
\end{enumerate}

{\realisation} 
Работа выполнена в Санкт-Петербургском государственном электротехническом университете «ЛЭТИ» имени В.И. Ульянова (Ленина). Основные теоретические и практические результаты диссертационного исследования внедрены в практику испытаний оптических систем в в филиале АО \flqq Корпорация \glqq Комета\grqq -- \glqq НПЦ ОЭКН\grqq \frqq при выполнении государственного контракта №099-К260/21/23 тема \flqq Разработка технологии изготовления и испытаний прецизионных зеркальных сканирующих оптико-механических систем для оптико-электронной аппаратуры космического базирования на основе многоядерных крупноформатных фотоприемных устройств\frqq.


{\probation}
Основные результаты работы докладывались и обсуждались на следующих конференциях:
\begin{enumerate}
	\item XV Международная конференция «Прикладная оптика–2022» 2022 Санкт-Петербург, Россия.
	\item XХV Всероссийская научно-техническая конференция молодых учёных «Навигация и управление движением» 2023 Санкт-Петербург, Россия.
	\item VI Научно-техническая конференция молодых ученых и специалистов «Будущее предприятия – в творчестве молодых» 2024 г. Санкт-Петербург. Россия.
\end{enumerate}

{\publications} Основные результаты по теме диссертации изложены
в~6~печатных изданиях,
3 из которых изданы в журналах, рекомендованных ВАК,
3 "--- в тезисах докладов.

\underline{\textbf{Объем и структура работы}}. Диссертация состоит из введения, 4 глав,
заключения и 1 приложения. Полный объём диссертации составляет 102 страницы, включая 34 рисунка и 6 таблиц. Список литературы содержит 73 наименования

\ifnumequal{\value{bibliosel}}{0}
{%%% Реализация пакетом biblatex через движок biber
    \begin{refsection}[bl-author, bl-registered]
        % Это refsection=1.
        % Процитированные здесь работы:
        %  * подсчитываются, для автоматического составления фразы "Основные результаты ..."
        %  * попадают в авторскую библиографию, при usefootcite==0 и стиле `\insertbiblioauthor` или `\insertbiblioauthorgrouped`
        %  * нумеруются там в зависимости от порядка команд `\printbibliography` в этом разделе.
        %  * при использовании `\insertbiblioauthorgrouped`, порядок команд `\printbibliography` в нём должен быть тем же (см. biblio/biblatex.tex)
        %
        % Невидимый библиографический список для подсчёта количества публикаций:
        \phantom{\printbibliography[heading=nobibheading, section=1, env=countauthorvak,          keyword=biblioauthorvak]%
        \printbibliography[heading=nobibheading, section=1, env=countauthorwos,          keyword=biblioauthorwos]%
        \printbibliography[heading=nobibheading, section=1, env=countauthorscopus,       keyword=biblioauthorscopus]%
        \printbibliography[heading=nobibheading, section=1, env=countauthorconf,         keyword=biblioauthorconf]%
        \printbibliography[heading=nobibheading, section=1, env=countauthorother,        keyword=biblioauthorother]%
        \printbibliography[heading=nobibheading, section=1, env=countregistered,         keyword=biblioregistered]%
        \printbibliography[heading=nobibheading, section=1, env=countauthorpatent,       keyword=biblioauthorpatent]%
        \printbibliography[heading=nobibheading, section=1, env=countauthorprogram,      keyword=biblioauthorprogram]%
        \printbibliography[heading=nobibheading, section=1, env=countauthor,             keyword=biblioauthor]%
        \printbibliography[heading=nobibheading, section=1, env=countauthorvakscopuswos, filter=vakscopuswos]%
        \printbibliography[heading=nobibheading, section=1, env=countauthorscopuswos,    filter=scopuswos]}%
        %
        \nocite{*}%
        %
        {\publications} Основные результаты по теме диссертации изложены в~\arabic{citeauthor}~печатных изданиях,
        \arabic{citeauthorvak} из которых изданы в журналах, рекомендованных ВАК%
        \ifnum \value{citeauthorscopuswos}>0%
            , \arabic{citeauthorscopuswos} "--- в~периодических научных журналах, индексируемых Web of~Science и Scopus%
        \fi%
        \ifnum \value{citeauthorconf}>0%
            , \arabic{citeauthorconf} "--- в~тезисах докладов.
        \else%
            .
        \fi%
        \ifnum \value{citeregistered}=1%
            \ifnum \value{citeauthorpatent}=1%
                Зарегистрирован \arabic{citeauthorpatent} патент.
            \fi%
            \ifnum \value{citeauthorprogram}=1%
                Зарегистрирована \arabic{citeauthorprogram} программа для ЭВМ.
            \fi%
        \fi%
        \ifnum \value{citeregistered}>1%
            Зарегистрированы\ %
            \ifnum \value{citeauthorpatent}>0%
            \formbytotal{citeauthorpatent}{патент}{}{а}{}%
            \ifnum \value{citeauthorprogram}=0 . \else \ и~\fi%
            \fi%
            \ifnum \value{citeauthorprogram}>0%
            \formbytotal{citeauthorprogram}{программ}{а}{ы}{} для ЭВМ.
            \fi%
        \fi%
        % К публикациям, в которых излагаются основные научные результаты диссертации на соискание учёной
        % степени, в рецензируемых изданиях приравниваются патенты на изобретения, патенты (свидетельства) на
        % полезную модель, патенты на промышленный образец, патенты на селекционные достижения, свидетельства
        % на программу для электронных вычислительных машин, базу данных, топологию интегральных микросхем,
        % зарегистрированные в установленном порядке.(в ред. Постановления Правительства РФ от 21.04.2016 N 335)
    \end{refsection}%
    \begin{refsection}[bl-author, bl-registered]
        % Это refsection=2.
        % Процитированные здесь работы:
        %  * попадают в авторскую библиографию, при usefootcite==0 и стиле `\insertbiblioauthorimportant`.
        %  * ни на что не влияют в противном случае
        \nocite{vakbib2}%vak
        \nocite{patbib1}%patent
        \nocite{progbib1}%program
        \nocite{bib1}%other
        \nocite{confbib1}%conf
    \end{refsection}%
        %
        % Всё, что вне этих двух refsection, это refsection=0,
        %  * для диссертации - это нормальные ссылки, попадающие в обычную библиографию
        %  * для автореферата:
        %     * при usefootcite==0, ссылка корректно сработает только для источника из `external.bib`. Для своих работ --- напечатает "[0]" (и даже Warning не вылезет).
        %     * при usefootcite==1, ссылка сработает нормально. В авторской библиографии будут только процитированные в refsection=0 работы.
}


