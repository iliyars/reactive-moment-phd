\chapter*{Заключение}                       % Заголовок
\addcontentsline{toc}{chapter}{Заключение}  % Добавляем его в оглавление

%% Согласно ГОСТ Р 7.0.11-2011:
%% 5.3.3 В заключении диссертации излагают итоги выполненного исследования, рекомендации, перспективы дальнейшей разработки темы.
%% 9.2.3 В заключении автореферата диссертации излагают итоги данного исследования, рекомендации и перспективы дальнейшей разработки темы.
%% Поэтому имеет смысл сделать эту часть общей и загрузить из одного файла в автореферат и в диссертацию:

Анализ существующих методов оценки влияния реактивных моментов показал необходимость создания специализированного стенда, обеспечивающего прямое измерение возмущающего момента, возникающего при повороте оптико-механических систем. Отсутствие подобных средств не позволяло верифицировать расчётные модели и надёжно учитывать динамические факторы, оказывающие влияние на качество изображения, формируемого оптической системой космического аппарата.

В диссертационной работе разработан и создан стенд, предназначенный для наземной оценки динамических воздействий подвижной оптико-механической системы. Он обеспечивает регистрацию реактивного момента, возникающего при повороте оптической нагрузки. Использование данного стенда дало возможность сопоставить расчётные и экспериментальные данные, подтвердить достоверность модели и выявить факторы, определяющие уровень остаточного реактивного момента.

В ходе проведённых испытаний было установлено, что величина реактивного момента может быть снижена за счёт эмпирического подбора инерционных характеристик компенсационных маховиков, а также выбора закона разгона привода. Экспериментально подтверждено, что применение синусоидального профиля движения позволяет уменьшить возбуждение высокочастотных колебаний и снизить уровень остаточного реактивного момента. 

Проведённый анализ динамических колебаний космического аппарата показал, что предложенные методы компенсации, отработанные на наземном стенде, обеспечивают выполнение требований к качеству оптического изображения в условиях реального функционирования. Таким образом, разработанная методика подтверждает свою практическую применимость и может быть использована не только для исследуемой оптико-механической системы, но и для настройки и испытаний других оптических систем с подвижными элементами.

Сформулированная цель исследования достигнута. Решены все поставленные в работе задачи, что позволило получить новые научные результаты в области оценки и компенсации реактивных моментов оптико-механических систем. Результаты исследования обладают как теоретической, так и практической значимостью: они расширяют представления о динамических свойствах оптико-механических систем и могут быть использованы при проектировании и испытаниях высокоточных систем наблюдения. Тем самым решена задача, имеющая важное значение для развития отечественного космического приборостроения и совершенствования систем стабилизации наблюдательных платформ.